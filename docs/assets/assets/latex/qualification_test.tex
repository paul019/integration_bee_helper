\documentclass[parskip=full]{article}

% Import packages
\usepackage[ngerman]{babel}
\usepackage[hidelinks]{hyperref}
\usepackage{multicol}
\usepackage{amsmath,amssymb,amsfonts}
\usepackage{amsthm}
\usepackage{mathrsfs}
\usepackage{graphicx}

% Set layout params
\usepackage[a4paper, top=1.5cm, bottom=1.5cm]{geometry}
\setlength\parindent{0pt}
\pagestyle{empty}

\newcommand{\dd}[1]{\text{d}{#1}}

% Arguments:
% 1. Competitor name
\newcommand{\coverSheet}[1]{
    \topskip0pt
    
    \vspace*{\fill}
    \begin{center}
        \textbf{Bitte erst auf Anweisung umdrehen!}
    \end{center}
    \vspace*{\fill}
    \begin{center}
        \Large
        Heidelberg Integration Bee 2024

        \Huge
        \textbf{Qualifikations-Test}

        \normalsize
    \end{center}
    \vspace*{\fill}
    \begin{center}
        \Large
        \textbf{#1}

        \normalsize
    \end{center}
    \vspace*{\fill}
    \textbf{Hinweise -- bitte gründlich durchlesen:}
    \begin{itemize}
        \item Dieser Qualifikations-Test enthält 20 Aufgaben, die innerhalb von 20 Minuten gelöst werden dürfen.
        \item Bei den Aufgaben handelt es sich um bestimmte Integrale -- die Lösungen sind also nur Zahlenwerte.
        \item Die Aufgaben sind bewusst so zusammengestellt, dass vermutlich nicht alle innerhalb der vorgegebenen Zeit gelöst werden können.
        \item Jede Aufgabe zählt gleich viel; es gibt genau dann einen Punkt auf eine Aufgabe, wenn die Lösung korrekt ist.
        \item Jede Lösung muss eingekreist werden. Es werden nur eingekreiste Lösungen gewertet. Ist mehr als eine Lösung eingekreist, so kann kein Punkt vergeben werden.
        \item Falls es nicht offensichtlich ist, schreibt an eine Lösung die Nummer der zugehörigen Aufgabe.
        \item Der Test darf erst geöffnet werden, wenn die Jury den Test startet. Sobald die Zeit abgelaufen ist, müssen alle Teilnehmenden sofort aufhören, zu schreiben.
    \end{itemize}
    \vspace*{\fill}
    \begin{center}
        \textbf{Viel Erfolg!}
    \end{center}
    \vspace*{\fill}

    \newpage
}

% Arguments:
% 1. Index
% 2. Latex
% 3. Name
\newcommand{\singlet}[3]{

    \begin{minipage}[t][8.6cm][t]{\textwidth}
        \textbf{#1. Aufgabe} \textit{#3}

        \begin{align*}
            #2
        \end{align*}
        
        \vspace*{\fill}

        \hfill\boxed{$\quad$/1}

        \vspace*{0.1cm}
    \end{minipage}%

}

% Arguments:
% 1. Total number of achievable points
\newcommand{\points}[1]{
    \vspace*{5.9cm}

    \hfill\boxed{$\quad$/#1}

    \newpage
}


\begin{document}
% \coverSheet{Martina Musterfrau}

% \hrulefill
% \singlet{1}{\int{x\dd{x}}}{Ergebnis vollständig vereinfachen!}
% \hrulefill
% \singlet{2}{\int{x\dd{x}}}{}
% \hrulefill
% \singlet{3}{\int{x\dd{x}}}{}
% \hrulefill
% \newpage

% \hrulefill
% \singlet{4}{\int{x\dd{x}}}{}
% \hrulefill
% \singlet{5}{\int{x\dd{x}}}{}
% \hrulefill
% \singlet{6}{\int{x\dd{x}}}{}
% \hrulefill
% \newpage

% \hrulefill
% \singlet{7}{\int{x\dd{x}}}{}
% \hrulefill
% \singlet{8}{\int{x\dd{x}}}{}
% \hrulefill
% \singlet{9}{\int{x\dd{x}}}{}
% \hrulefill
% \newpage

% \hrulefill
% \singlet{10}{\int{x\dd{x}}}{}
% \hrulefill
% \singlet{11}{\int{x\dd{x}}}{}
% \hrulefill
% \singlet{12}{\int{x\dd{x}}}{}
% \hrulefill
% \newpage

% \hrulefill
% \singlet{13}{\int{x\dd{x}}}{}
% \hrulefill
% \singlet{14}{\int{x\dd{x}}}{}
% \hrulefill
% \singlet{15}{\int{x\dd{x}}}{}
% \hrulefill
% \newpage

% \hrulefill
% \singlet{16}{\int{x\dd{x}}}{}
% \hrulefill
% \singlet{17}{\int{x\dd{x}}}{}
% \hrulefill
% \singlet{18}{\int{x\dd{x}}}{}
% \hrulefill
% \newpage

% \hrulefill
% \singlet{19}{\int{x\dd{x}}}{}
% \hrulefill
% \singlet{20}{\int{x\dd{x}}}{}
% \hrulefill
% \points{20}

% PLACEHOLDER
\end{document}